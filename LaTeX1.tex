\documentclass[11pt]{article} 
\usepackage{amssymb,amsfonts,amsmath,amsthm} 
\usepackage[polish]{babel} 
\usepackage[cp1250]{inputenc} 
\usepackage[T1]{fontenc}

\begin{document}

Hello World! To oczywimcie tylko zabawa.

A to jest drugi akapit i można zobaczyć efekt działania.
%using \\ for line breaks
Hello World! To oczywiście\\ tylko zabawa.

A to jest drugi akapit i można \linebreak zobaczyć efekt działania.

A to jest kolejny akapit i można \newline zobaczyć efekt działania.

%changing the font size
Czasami zależy nam na zmianie stopnia pisma. Lubię {\large duże}, {\Large
większe} oraz {\LARGE bardzo duże} litery.

%changing the font style
Odmiń pochyłe składa się terminy definiowane, objamniane lub tłumaczone. %Vlad: Tu ni mogę odtworzyć

Grzegorz ugotował {\em knedle}, rodzaj {\bf pulpetów} z surowego \underline{mięsa}

%justifications (omówiła srodowiska str 31) 
Wyrównanie tekstu wymaga „pracy":
\begin{center}środek \end{center}
\begin{flushright}do prawa\end{flushright}

To samo tylko jeszcze lewo:
\begin{flushleft}do lewa \end{flushleft} 
\begin{center}środek \end{center}
\begin{flushright}do prawa\end{flushright}

%quoting
Cudzysłowy pojawią sie jeżeli je podwoimy ''BASIA'', pojedyńcze dają mmieszny efekt 'BARBARA', a takie "Basia" \ świadczą podobno o typograficznym... .

Efekt specjalny, jeżeli chcemy coś wyróznić w środowisku, to wtedy środowisko
 
\verb+ quote+
\begin{quote}
bardzo ładna poezja bardzo ładna poezja bardzo ładna poezja 
bardzo ładna poezja bardzo ładna poezja bardzo ładna poezja 
bardzo ładna poezja bardzo ładna poezja bardzo ładna poezja 
bardzo ładna poezja bardzo ładna poezja \end{quote}

Tutaj należy skończyć stronę (znależć jak).
\newpage
%other tricks (środowiska itemize, enumerate i description)

\vspace*{-3cm}
Teraz pobawimy sie listami;\\ środowiska \verb+ enumerate, itemize, description+,\\
„spróbować" osiągnąć efekt jak poniżej:
\begin{enumerate}\item Taka lista:
\begin{itemize}
\item wygląda
\item[--] śmiesznie. \end{itemize}
\item Pamiętaj:
\begin{description}
\item[Głupoty] nie stają się mądrościami, gdy sie je wyliczy. 
\item[Mądrości] można elegancko zestawiać w wyliczeniach \end{description}
\end{enumerate}
%mmm. math mode.

I wreszcie „matematyka", które należy po prostu napisać\\ 
Równanie ($f(x)=2x$) można zapisać :
\[ f(x)=2x \]
\begin{equation}
f(x)=2x
\end{equation}

%the big secret of LaTeX
$$\Gamma(\gamma)\neq 1\rightarrow \exists \hbar \forall \heartsuit $$

%subscripts, superscripts 
$$z_{1} = x^{22} > 2^{2^2}$$
%integrals, summations
$$ \sum_{i=a}^b F(x) \Delta x \approx \int_a^b f(x) dx $$ 
bardzo $ \sum_{i=a}^b F(x) \Delta x \approx \int_a^b f(x) dx $
ładna poezja bardzo ładna poezja bardzo ładna poezja bardzo ładna poezja bardzo ładna poezja bardzo

$$\frac{a+b}{c-d}$$ 
$$\sqrt[3]{\frac{a+b}{c-d}$$
$$f'(x)=2x \hspace{2em} \Rightarrow \hspace{2em} f(x)=x^2+C$$

środowisko \verb+{array}+ do tworzenia tabel i macierzy:

pierwsza
$$
\begin{array}{c1r}
1 & 22 & 3 \\ 
99 & 5 & x^2 
\end{array}
$$

druga
$$
\begin{array}{c1r}
\hline
1 & 22 & 3 \\ 
\hline
99 & 5 & x^2 
\end{array} $$

trzecia
$$
\left[
\left(
\begin{array}{c1r}
1 & 22 & 3 \\
99 & 5 & x^2 
\end{array} 
\right)
+\frac{1}{2} 
\right]
$$
 
\end{document}
