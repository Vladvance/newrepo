\documentclass[12pt]{article} 
\usepackage{amsmath}
\usepackage[polish]{babel}
\usepackage[cp1250]{inputenc}
\usepackage[OT4]{fontenc} 
\usepackage{setspace}
\title{Tytuł: Matematyka}
\date{} %bez tej linijki, pod tytułem wypisze się dzisiejsza 
%data; ewentualnie w nawiasy możemy wstawić jakąś konkretną datę,
%a jeśli w nawiasach nic nie bedzie to data w ogóle sie nie wyświetli

\begin{document}
\maketitle %bez tego tytuł sie nie bedzie wyświetlał 
\section*{Ściąga} %gwiazdka powoduje, ze tak sekcja nie ma numerka, i nie pojawia się w spisie treści
int,lim,frac,sum,lnot,land,implies,Rightarrow,infty,iff,nearrow

\section{Trygonometria (array)}
%nie robic poniższej tabelki jako \begin {table} \begin{tabular} 
%bo wtedy kazdy %wzorek z każdej komórki tabelki trzeba pisać w dolarach.
%Jak sie skorzysta ze srodowiska {array} ujmując je cale w podwójne dolary albo %\[ i \] 
%to od razu wszystko co będzie w tej tabeli będzie jako matematyka
%jeżeli chcemy żeby tabelka miała więcej "światła" w pionie użyjmy pakietu
%setspace.sty i środowiska \begin{spacing}{2} \end{spacing} \begin{spacing}{1,5}
{\[ \begin{array}{|c|c|c|c|c|c||c|c|c|c|rr}
  \hline	% robi poziome kreski w tabelach
  x & 0 & \frac{\pi}{6} & \frac{\pi}{4} & \frac{\pi}{3} & \frac{\pi}{2} & (0;\frac{\pi}{2}) & (\frac{\pi}{2};\pi) & (\pi; \frac{3\pi}{2}) & (\frac{3\pi}{2};2\pi) \\
  \hline
  \sin x & O & \frac{1}{2} & \frac{\sqrt{2}}{2} & \frac{\sqrt{3}}{2} & 1 & + & + & - & - \\
  \hline
\end{array} \]}
\end{spacing}
\section{Szeregi (description)}
  \begin{description}
    \item[Szereg nieskończony:] $\displaystyle\sum_{n=1}^\infty a_{n} = (a_{n},S_{n})$ %bez \displaystyle symbol sumowanie bedzie tekstowo wypisany
    \item[Szereg jest zbieżny:] $\displaystyle\sum_{n=1}^\infty a_{n} < \infty $
    \item[Szereg jest rozbieżny do nieskończoności:] $\displaystyle\sum_{n=1}^\infty a_{n} = \infty $
  \end{description}
  \noindent Warunek konieczny zbieżności: $ \sum_{n=1}^\infty a_{n} < \infty \Rightarrow \lim_{n\rightarrow\infty} a_n = 0$ \\ %podwojny backslash lamie linie
  (wniosek: $ \displaystyle \lim_{n\rightarrow\infty} a_n \neq O \Rightarrow \sum_{n=1}^\infty = \infty$) \\
  $ \displaystyle\sum_{n=1}^\infty a_n < \infty $ jest bezwzględnie zbieżny,	$ \displaystyle\ sum_{n=1}^\infty 
a_nl < \infty $, w przeciwnym wypadku jest warunkowo zbieżny.

  \subsection{Szeregi o wyrazach dodatnich (equation,tag)} 
    \subsubsection{Funkcja Riemanna: (cases,łext)}
    
    \begin{equation}
    % srodowisko equation ponumeruje wzory (zeby wymusic konkretny numerek np. O wpisujemy \tag{0}
    % jak nie chcemy mieć numerowania to wzór ujmujemy w podwójne dolary,
    %jak chcemy mieć wzór bezpośrednio w tekście to w pojedyńcze dolary
    \zeta(s) = \sum_{n=1}^\infty \frac{1}{n^s} = \begin{cases} 
      \infty & \text{dla s} \leq 1 \\ 
      -\infty & \text{dla s} > 1
    \end{cases}\tag{0}
      %zeby ujmować coś w klamre jako kolejne opcje korzystamy ze środowiska {cases} 
      %i poszczegolne elementy oddzielamy nową linią czyli podwójnym backslashem
      %zeby jakiś fragment byl pisany jako tekst a nie jako wzor to trzeba go ujac w \tekst{...}
    \end{equation}

  \subsection{Szeregi o wyrazach dowolnych (align)}
    \subsubsection{Kryterium Abela:}
    \begin{align} % align chyba też robi wzóry z numerkami, wzóry powinny być wyrównane
      \left(\lnot\left[\left(a_{n}\right)\nearrow\right]\land\forall n \in N : \left(a_{n} 0\right)\land \sum\limits_{n=i}^{\infty}
      b_{n} < \infty \right)\Rightarrow\sum\limits_{n=1}^{\infty}\left(a_{n}b_{n}\right) < \infty
    \end{align}
    
    \subsubsection{Kryterium Dirichleta:}

    \begin{align}
      \left(\lim\limits {n\rightarrow\infty}a_{n} = 0 \land \lnot \left[ \left(a_{n} \right) \nearrow \right] \land \exists \epsilon > O :
      \forall n \in N : \epsilon - \left|S_{n}\right| O \right) \nonumber\\
      \Rightarrow \sum\limits_{n=1}^\infty} \left(a_{n}b_{n} \right) < \infty
    \end{align}

    \subsubsection{Kryterium Leibniza:} 
    
    \begin{align}
      \lim\limits_{n \rightarrow \infty} a_{n} = O \iff \sum\limits_{n=1}^{\infty} \left( \left(-1\right)^{n} a_{n}\right) < \infty
    \end{align}

\section{Całki (gather,int)}
\begin{gather} %gather też robi wzóry z numerkami, ale wyśrodkowane;
  \int\frac{f'(x)}{f(x)}dx = \ln\left|f(x)\right|	\\
  \int xdx = x^{2} + C \\
\int \cos xdx = \sin x + C
\end{gather}

\section{Kombinatoryka (align*, \{a $\backslash$choose b\})} 
\begin{align*} % align* robi wzórki wyrównane ale bez numerowania
{n \choose k} = \frac{n!}{(n-k)!k!}\\
{n \choose n-k} - {n \choose k}
\end{align*}

\tableofcontents %polecenie robiące spis tresci, spis jest w osobnym pliku o rozszerzeniu .toc (table of contents)
\end{document}
